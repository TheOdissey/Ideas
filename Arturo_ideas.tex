%%%%%%%%%%%%%%%%%%%%%%%%%%%%%%%%%%%%%%%%%
% Large Colored Title Article
% LaTeX Template
% Version 1.1 (25/11/12)
%
% This template has been downloaded from:
% http://www.LaTeXTemplates.com
%
% Original author:
% Frits Wenneker (http://www.howtotex.com)
%
% Modified by:
% Arturo Pacifico Griffini
%
% License:
% CC BY-NC-SA 3.0 (http://creativecommons.org/licenses/by-nc-sa/3.0/)
%
%%%%%%%%%%%%%%%%%%%%%%%%%%%%%%%%%%%%%%%%%

%----------------------------------------------------------------------------------------
%	PACKAGES AND OTHER DOCUMENT CONFIGURATIONS
%----------------------------------------------------------------------------------------

\documentclass[DIV=calc, paper=a4, fontsize=11pt, twocolumn]{scrartcl}	 % A4 paper and 11pt font size

\usepackage{lipsum} % Used for inserting dummy 'Lorem ipsum' text into the template
\usepackage[english]{babel} % English language/hyphenation
\usepackage[protrusion=true,expansion=true]{microtype} % Better typography
\usepackage{amsmath,amsfonts,amsthm} % Math packages
\usepackage[svgnames]{xcolor} % Enabling colors by their 'svgnames'
\usepackage[hang, small,labelfont=bf,up,textfont=it,up]{caption} % Custom captions under/above floats in tables or figures
\usepackage{booktabs} % Horizontal rules in tables
\usepackage{fix-cm}	 % Custom font sizes - used for the initial letter in the document

\usepackage{hyperref} %for hypertext links
\hypersetup{
    bookmarks=true,         % show bookmarks bar?
    unicode=false,          % non-Latin characters in Acrobat�s bookmarks
    colorlinks=true,       % false: boxed links; true: colored links
    linkcolor=black,          % color of internal links (change box color with linkbordercolor)
    citecolor=green,        % color of links to bibliography
    filecolor=magenta,      % color of file links
    urlcolor=blue           % color of external links
}

\usepackage{sectsty} % Enables custom section titles
\allsectionsfont{\usefont{OT1}{phv}{b}{n}} % Change the font of all section commands

\usepackage{fancyhdr} % Needed to define custom headers/footers
\pagestyle{fancy} % Enables the custom headers/footers
\usepackage{lastpage} % Used to determine the number of pages in the document (for "Page X of Total")

% Headers - all currently empty
\lhead{}
\chead{}
\rhead{}

% Footers
\lfoot{}
\cfoot{}
\rfoot{\footnotesize Page \thepage\ of \pageref{LastPage}} % "Page 1 of 2"

\renewcommand{\headrulewidth}{0.0pt} % No header rule
\renewcommand{\footrulewidth}{0.4pt} % Thin footer rule

\usepackage{lettrine} % Package to accentuate the first letter of the text
\newcommand{\initial}[1]{ % Defines the command and style for the first letter
\lettrine[lines=3,lhang=0.3,nindent=0em]{
\color{DarkGoldenrod}
{\textsf{#1}}}{}}

%----------------------------------------------------------------------------------------
%	TITLE SECTION
%----------------------------------------------------------------------------------------

\usepackage{titling} % Allows custom title configuration

\newcommand{\HorRule}{\color{DarkGoldenrod} \rule{\linewidth}{1pt}} % Defines the gold horizontal rule around the title

\pretitle{\vspace{-30pt} \begin{flushleft} \HorRule \fontsize{50}{50} \usefont{OT1}{phv}{b}{n} \color{DarkRed} \selectfont} % Horizontal rule before the title

\title{5 Startup Ideas} % Your article title

\posttitle{\par\end{flushleft}\vskip 0.5em} % Whitespace under the title

\preauthor{\begin{flushleft}\large \lineskip 0.5em \usefont{OT1}{phv}{b}{sl} \color{DarkRed}} % Author font configuration

\author{Arturo Pacifico Griffini, } % Your name

\postauthor{\footnotesize \usefont{OT1}{phv}{m}{sl} \color{Black} % Configuration for the institution name
University of California Berkeley% Your institution

\par\end{flushleft}\HorRule} % Horizontal rule after the title

\date{} % Add a date here if you would like one to appear underneath the title block

%----------------------------------------------------------------------------------------

\begin{document}

\maketitle % Print the title

\thispagestyle{fancy} % Enabling the custom headers/footers for the first page 

%----------------------------------------------------------------------------------------
%	ABSTRACT
%----------------------------------------------------------------------------------------

% The first character should be within \initial{}
\initial{T}\textbf{his is a short proposal on five possible startups that can developed by Odissey. For each startup, I shortly describe the problem it is trying to solve, the way it could solve it, who wants it, who is doing it already, how to get paid for it, and a list of technologies that I believe would be required other than the usual web development technologies. }

%----------------------------------------------------------------------------------------
%	ARTICLE CONTENTS
%----------------------------------------------------------------------------------------

\section*{SkillEx}
SkillEx is a web platform that allows users to share and sell their skills to other users. Young men and women that have a particular strong interest and consequent skills\footnote{For example a 'skill/expertise' could be skateboarding, skiing, juggling, pen tricks, photography, painting, etc...} desire to meet people with similar skills to practice together and share expertise. Also they might be interested to teach in exchange of learning some other skill or  of monetary compensation. Hence the platform would focus on the following three user stories:
\begin{enumerate}
\item As a passionate expert, I want to find and meet other passionate experts in the same field as mine to practice and form a community of people that share a passion.
\item As an expert I would like to teach other people my expertise in exchange of learning another skill or monetary compensation.
\item As a young man/woman I want to learn a new skill from an expert in my area, because I can't find a business that can teach me what I want to learn and I want an informal friendly setting.
\end{enumerate} 
The 'experts' will have a public profile, and will prove they have the skills they claim by uploading a video of themselves. We would take care of the transactions and take a small commission on them.

I could not find anyone doing anything similar online and I honestly how large is the target market. However, I love features (1) and (3), and would use this service. The MVP could start with (1) and develop in the rest once we have a good network of people. We need: video player (for the videos), location tracker (to automatically chase the location of the 'experts'), and some payment function.

I heard the name SkillEx at startup weekend for a similar platform and really liked it.

\section*{WeMakeIt}
WeMakeIt is a web platform that allows freelance product designers to sell their products without having to worry about manufacturing. The freelance designers would upload their product on our site and we would asses the production cost, decide on a sale price, and how many units is profitable for us to sell (break even * 1.x). Then visitors of the site can choose the cool products they like and buy them. If we presell the amount of units we decided in advance within a certain time limit (say one month), we manufacture the product and ship it. Else, we reimburse the people that already payed for it. FInally, for each sold and shipped product, we would then send a large share of the profit (say 70\%) to the product designers and keep the rest. In a way, it would feel like Kickstarter, but we would take care of the production overhead. We could start by printing t-shirts, and search for attractive product categories. The MVP could focus on the following user stories:
\begin{enumerate}
\item As a college student with an idea for some cool t-shirt and a skill in adobe illustrator or alike, I want to sell the t-shirt I designed, because I would not mind some extra cash and I would find pride in seeing students with my t-shirt walking around campus.
\item As a man/woman looking for something original, I am looking to buy some  products that are close to unique (not many units sold), because I want to differentiate myself from the crowd I have something that you would not usually find. 
\end{enumerate}
We would need to find a product category for which we could easily delegate the manufacturing, and the marginal cost for manufacturing the products would be low. The best I can think of right now is t-shirts.

I have heard this idea from a business guy a couple of weeks ago and unfortunately I found a site that is doing this already, \href{http://www.unire.us/}{unire}. Most likely the site is from the guy that told me about the idea. However, I believe we could do it much better than the way he is doing it right now.



\section*{TravelBox}
A web platform that prepares for city visitors a package of tickets for events that are happening during the time the tourist is visiting the city. Travellers do not know what to do most of time when they are visiting a city and spend a lot of time trying to figure out what to do in that city. What if you could receive of package of tickets of events that are happening while you are there, by just providing your flight reservation number and by selecting some categories of events that my be of interest to you? We would take out the overhead of figuring out what to do, book the event, and the only thing the traveller would need to do would be clicking yes or no for each event, or event group. There is a wide range of startups that is tackling individual aspects of this idea. Like recommending the events to you, or book the tickets for you, but no one is tackling the entire package. We need some aggregator technology and perhaps some machine learning to make better recommendation. It is not an ideal idea for the constraints we have this semester, but is something interesting to think about.

\section*{Unplugged}
A web platform for party organisers to rent out amateur music bands that have no agent. This idea targets men and women that have a passion for performing arts and are not professionals in what they are passionate about, and event organisers that are looking for performances for their events. The platform would focus on the following user stories:
\begin{enumerate}
\item As a person that performs music regularly with a group of friends in my free time, I want to play at live events, because I want to earn little money and I find pride in performing in front of strangers and be payed for it.
\item As an event organiser, I want to hire a music band for a live performance without going through an agent, because I do not want to spend a lot of money and still be able to enjoy atmosphere created by live performances. 
\end{enumerate}
The performers would create a profile for their group, which would specify where they are willing to perform, in what category do they fall into, and for what kind of occasion would they be ready and willing to perform. Then, they would upload a video of their performance(s) to prove their skill set and ask for an hourly rate. The event organisers can then search through all this people and find the ones that best fit their interest for a great price. performers and hire the ones they like. \href{http://www.sonicbids.com/}{SonicBids} is doing something similar, but focuses on semi-professional music bands that have an agent. We would focus on amateurs ranging from music, to dance, to magic.
We could take care of the transaction process between the event organisers and the performers and take a commission on every transaction. We could also organising promotion events, where the bands and performers with the best reviews would perform for interested production companies. I would not need any special technology.


\section*{NeverIdle}
A web platform that allows owners of machinery that is idle most of the time to sell out the machinery usage to people that live within a certain distance from them. For example I want to buy and use a 3D printer, but I am looking of some way to make the purchase more affordable. I know that I am not going to use the printer 100\% of the time. Hence, I could rent out its usage to amortise its price or even make some profit out of it. In a way, we would be making the laundry centre business model available to every machinery owner. This could be an incentive for some people to buy machinery like 3D printers as an investment, since we would provide a platform to easily market their machinery. The startup would focus on the following two user stories: 
\begin{enumerate}
\item As a machinery owner, I want to rent out the usage of my machine during its idle time, because I want to make some extra money out of it.
\item As a person looking to use a specific machine, I prefer to rent out the usage of the machine instead of buying it, because of my financial limitations.
\end{enumerate}

At \href{http://www.3dreambox.com/}{3DreamBox} they are providing 3D printers that can be used by anyone at an usage fee. We would not be limited by one type of machinery and reach a much bigger market. We would not require any particular technology. Revenue streams could be generated by commission on usage and machine promotion once the market becomes fairly competitive. I believe that we will face the main challenges when we will acquiring  the first 'machine database' and when we will be managing the user experience from a safety perspective for both the machinery owner and the user.


%----------------------------------------------------------------------------------------
%	REFERENCE LIST
%----------------------------------------------------------------------------------------

%\begin{thebibliography}{99} % Bibliography - this is intentionally simple in this template
%
%\bibitem[Figueredo and Wolf, 2009]{Figueredo:2009dg}
%Figueredo, A.~J. and Wolf, P. S.~A. (2009).
%\newblock Assortative pairing and life history strategy - a cross-cultural
%  study.
%\newblock {\em Human Nature}, 20:317--330.
% 
%\end{thebibliography}

%----------------------------------------------------------------------------------------

\end{document}


